% ----------------------------------------------------------------------------
% ------------------------Bachelor Thesis
% ----------------------------------------------------------------------------
\documentclass[11pt]{beamer}
\usepackage[english,ngerman]{babel}
\usepackage[T1]{fontenc}
\usepackage[utf8]{inputenc}
\usepackage{lmodern,microtype,lipsum,csquotes,
	enumerate,amssymb,fixltx2e,listings,lastpage,fancybox}
\usepackage[scaled=.9]{inconsolata}
%\usepackage[nocut]{thmbox}
\usepackage{graphicx}
\usepackage{tikz}
\usepackage{listings}
\usetikzlibrary{shapes,snakes}

% -----------------------------------------------------
% ---------------- Package setup
% -----------------------------------------------------
%\hypersetup{pdfstartview={FitH}}

\lstdefinestyle{sparql} {
		language=sparql,
		emph=[1]{alpha,beta,zeta,gamma,epsilon,delta,tau,Person,Plane,City,TimeZone,Language,Country,Citizen,desc,media,title,author,
			age,name,person},
		emphstyle=[1]{\color{ScarletRed}}
		}
		
\lstdefinestyle{pscode} {
		language=c,
		emph=[1]{while,String,for,each,in,Iterator},
		emphstyle=[1]{\color{ScarletRed}}
		}
		
\lstdefinestyle{mpqf} {
		language=xml,
		emph=[1]{MpegQuery,Input,Query,OutputDescription,QueryCondition,Condition,SemanticRelation,SortBy,ReqField,
			GlobalComment,Description,ResultItem,Output,SystemMessage,Status,Code,ServiceID,DesiredCapability,Management,
			AvailableCapability,QFDeclaration,Prefix,SemanticField,Var,Subject,Property,Object,SemanticArithmeticField,
			LongValue,SPARQL,@prefix,sparql,head,results,result,uri,variable,binding,literal,rdf:RDF,rdf:Description,
			TargetMediaType,MediaResource,DateTimeField,DateValue,MediaUri,SupportedQFProfile,SupportedMetadata,
			SupportedExampleMediaTypes,SupportedResultMediaTypes,AbstractRDFXML,Node,Arc,ex:properties,ex:name,ex:address,
			ex:relationship,ex:belongsto,SemanticStringField,StringValue,ReqSemanticField},
		emphstyle=[1]{\color{SkyBlue}}
		}
		
\lstset{
		breakautoindent=true,
		breakindent=2em,
		breaklines=true,
		tabsize=4,
		frame=blrt,
		frameround=tttt,
		captionpos=b,
		basicstyle=\tiny\ttfamily,
		keywordstyle={\color{SkyBlue}},
		%commentstyle={\color{OliveGreen}},
		stringstyle={\color{ScarletRed}},
		showspaces=false,
		%numbers=right,
		%numberstyle=\scriptsize,
		%stepnumber=1, 
		%numbersep=5pt,
		%showtabs=false
		prebreak = \raisebox{0ex}[0ex][0ex]{\ensuremath{\hookleftarrow}},
		aboveskip={1.5\baselineskip},
		columns=fixed,
		upquote=true,
		extendedchars=true
		}

% ------------------------------------------
% -------- xcolor - (Tango)
% ------------------------------------------

\definecolor{LightButter}{rgb}{0.98,0.91,0.31}
\definecolor{LightOrange}{rgb}{0.98,0.68,0.24}
\definecolor{LightChocolate}{rgb}{0.91,0.72,0.43}
\definecolor{LightChameleon}{rgb}{0.54,0.88,0.20}
\definecolor{LightSkyBlue}{rgb}{0.45,0.62,0.81}
\definecolor{LightPlum}{rgb}{0.68,0.50,0.66}
\definecolor{LightScarletRed}{rgb}{0.93,0.16,0.16}
\definecolor{Butter}{rgb}{0.93,0.86,0.25}
\definecolor{Orange}{rgb}{0.96,0.47,0.00}
\definecolor{Chocolate}{rgb}{0.75,0.49,0.07}
\definecolor{Chameleon}{rgb}{0.45,0.82,0.09}
\definecolor{SkyBlue}{rgb}{0.20,0.39,0.64}
\definecolor{Plum}{rgb}{0.46,0.31,0.48}
\definecolor{ScarletRed}{rgb}{0.80,0.00,0.00}
\definecolor{DarkButter}{rgb}{0.77,0.62,0.00}
\definecolor{DarkOrange}{rgb}{0.80,0.36,0.00}
\definecolor{DarkChocolate}{rgb}{0.56,0.35,0.01}
\definecolor{DarkChameleon}{rgb}{0.30,0.60,0.02}
\definecolor{DarkSkyBlue}{rgb}{0.12,0.29,0.53}
\definecolor{DarkPlum}{rgb}{0.36,0.21,0.40}
\definecolor{DarkScarletRed}{rgb}{0.64,0.00,0.00}
\definecolor{Aluminium1}{rgb}{0.93,0.93,0.92}
\definecolor{Aluminium2}{rgb}{0.82,0.84,0.81}
\definecolor{Aluminium3}{rgb}{0.73,0.74,0.71}
\definecolor{Aluminium4}{rgb}{0.53,0.54,0.52}
\definecolor{Aluminium5}{rgb}{0.33,0.34,0.32}
\definecolor{Aluminium6}{rgb}{0.18,0.20,0.21}
\definecolor{Brown}{cmyk}{0,0.81,1,0.60}
\definecolor{OliveGreen}{cmyk}{0.64,0,0.95,0.40}
\definecolor{CadetBlue}{cmyk}{0.62,0.57,0.23,0}

% ------------------------------------------
% -------- misc
% ------------------------------------------
\renewcommand*\oldstylenums[1]{{\fontfamily{fxlj}\selectfont #1}}

\usetheme{CambridgeUS}
\useinnertheme{rectangles}
%dove | beaver  |orchid
\usecolortheme{dove} %albatross | beaver | beetle |crane | default | dolphin |dove | fly | lily | orchid |rose |seagull | seahorse |sidebartab | structure |whale | wolverine
%\AtBeginSection[]{\begin{frame}\frametitle{Table of Contents}\tableofcontents[currentsection]\end{frame}}
\beamertemplatenavigationsymbolsempty

% ------------------------------------------
% -------- hyphenation rules
% ------------------------------------------
\hyphenation{}

\begin{document}
%\newtheorem[L]{definition}{Definition}
\graphicspath{{img/}}

% Title stuff
\title[Multimedia Datenbanken]{Projekt 3}
\subtitle[Multimedia Datenbanken]{}
\author[Bastian Huber et al.]{Bastian Huber, Daniel Watzinger, Sebastian Rainer, Benedikt Preis}
\institute[Universität Passau]{Professur für Informatik mit Schwerpunkt Medieninformatik, Universität Passau}
\date{\today}

\frame{
	\titlepage
}

\begin{frame}
	\frametitle{Inhaltsverzeichnis}
	\tableofcontents
\end{frame}

\section{Color Structur Descriptor}
\subsection{ColorStructurDescriptor.java}
\begin{frame}
	\frametitle{Implementierung ColorStructurDescriptor.java}
	\begin{itemize}
		\item Grundlegende Fehler bei der Umsetzung des Standards
		\item Beispiele falls erwünscht im Anhang
		\item Fehler sind gravierend
		\item $\rightarrow$ Eigene Implementierung des Standards
	\end{itemize}
\end{frame}

\subsection{Eigene Implementierung des MPEG-7 Color Structur Descriptor}
\begin{frame}
	\frametitle{MPEG-7 Color Structur Descriptor}
	Grundlegende Schritte:
	\begin{itemize}
		\item HMMD Color Space Konvertierung
		\item Quantisieren des Color Space
		\item Aufsummiern der Quantisierten Farben
		\item Non-Uniform (mid-point) Quantisation der aufsummierten Werte
		\item Distanzmetrik: gewichtete $L^1$-Norm, optimiert für Human Recognition
		\item Requantisierung falls zwei CSDs mit unterschiedlicher Größe $\rightarrow$ Aufsummieren und erneutes Quantisieren
	\end{itemize}
\end{frame}


\subsection{Integration des CSD}
\begin{frame}
	\frametitle{Integration des CSD}
	\begin{itemize}
	  \item ColorStructurDescriptorImplementation.java (Eigene Implementierung des CSD)
	  \item ColorStructurDescriptor.java (Schnittstelle zu LIRE, delegiert an ColorStructurDescriptor.java)
	  \item Erweitern von DocumentBuilderFactory.java um den CSD LIRE zugänglich zu machen
	\end{itemize}
\end{frame}

\section{Bewertung}
\begin{frame}
	\frametitle{Bewertung der Features}
	\begin{itemize}
	  \item Auswahl eines Bildes $\rightarrow$ Query By Example
	  \item Subjektive Auswahl von relevanten/ähnlichen Bildern
	  \item Vergleich mit den Suchergebissen $\rightarrow$ Precision/Recall
	  \item Wang Datensatz 0-99: QBE (1.jpg) $\rightarrow$ Color Structur bzw. Scalable Color
	  \item Wang Datensatz 500–599: QBE (594.jpg ,568.jpg) $\rightarrow$ Color Structure Descriptor hohe Precision, Color Layout Descriptor hohen Recall
	\end{itemize}
\end{frame}


\section{Fragen \& Demo}
\begin{frame}
	\frametitle{Fragen und Demo}
\end{frame}


\begin{frame}
	\frametitle{Anhang I - Fehler in ColorStructurDescriptor.java}
	Falsche Größe des Bildes.
	\lstinputlisting[language=Java]{error_1.java}
\end{frame}

\begin{frame}
	\frametitle{Anhang II - Fehler in ColorStructurDescriptor.java}
	\scriptsize
	\lstinputlisting[language=Java]{error_2.java}
\end{frame}

\begin{frame}
	\frametitle{Anhang III - Fehler in ColorStructurDescriptor.java}
	\scriptsize
	\lstinputlisting[language=Java]{error_3.java}
\end{frame}

\begin{frame}
	\frametitle{Anhang  - Distanzmetrik}
		    Gewichtete Variante der $L^1$-Norm. Sie erzielt sehr gute Ergebnisse auf dem Gebiet der
    \emph{Human Recognition} und des \emph{Body Recognition}.
$$
  weight_{csd,csd'}(i) = \frac{csd(i) + csd'(i)}{\sum\limits_{k = 1}^N csd(k) + csd'(k)}
$$
$$
  distance(csd,csd') = \sum\limits_{i = 1}^N   weight_{csd,csd'}(i) \cdot |csd(i) - csd'(i)|
$$
\end{frame}

\end{document}
