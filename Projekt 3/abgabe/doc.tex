\documentclass{article}
\usepackage[T1]{fontenc}
\usepackage[ngerman,english]{babel}
\usepackage[latin1]{inputenc}
\usepackage{lmodern,calc,microtype,parskip,lipsum,booktabs,textcomp,csquotes,enumerate,amssymb,
  vmargin,fancyhdr,fixltx2e,makeidx,listings,ellipsis,remreset,xcolor,lastpage,caption,fancybox,verbatim,amsmath}
\usepackage{graphicx}
\usepackage[pdftex]{hyperref}
\usepackage{amsmath,chngcntr}
\usepackage{listings}

% Title
\def\thetitle{Multimedia Datenbanken --- Projekt }

% ------------------------------------------
% -------- xcolor - (Tango)
% ------------------------------------------

\definecolor{LightButter}{rgb}{0.98,0.91,0.31}
\definecolor{LightOrange}{rgb}{0.98,0.68,0.24}
\definecolor{LightChocolate}{rgb}{0.91,0.72,0.43}
\definecolor{LightChameleon}{rgb}{0.54,0.88,0.20}
\definecolor{LightSkyBlue}{rgb}{0.45,0.62,0.81}
\definecolor{LightPlum}{rgb}{0.68,0.50,0.66}
\definecolor{LightScarletRed}{rgb}{0.93,0.16,0.16}
\definecolor{Butter}{rgb}{0.93,0.86,0.25}
\definecolor{Orange}{rgb}{0.96,0.47,0.00}
\definecolor{Chocolate}{rgb}{0.75,0.49,0.07}
\definecolor{Chameleon}{rgb}{0.45,0.82,0.09}
\definecolor{SkyBlue}{rgb}{0.20,0.39,0.64}
\definecolor{Plum}{rgb}{0.46,0.31,0.48}
\definecolor{ScarletRed}{rgb}{0.80,0.00,0.00}
\definecolor{DarkButter}{rgb}{0.77,0.62,0.00}
\definecolor{DarkOrange}{rgb}{0.80,0.36,0.00}
\definecolor{DarkChocolate}{rgb}{0.56,0.35,0.01}
\definecolor{DarkChameleon}{rgb}{0.30,0.60,0.02}
\definecolor{DarkSkyBlue}{rgb}{0.12,0.29,0.53}
\definecolor{DarkPlum}{rgb}{0.36,0.21,0.40}
\definecolor{DarkScarletRed}{rgb}{0.64,0.00,0.00}
\definecolor{Aluminium1}{rgb}{0.93,0.93,0.92}
\definecolor{Aluminium2}{rgb}{0.82,0.84,0.81}
\definecolor{Aluminium3}{rgb}{0.73,0.74,0.71}
\definecolor{Aluminium4}{rgb}{0.53,0.54,0.52}
\definecolor{Aluminium5}{rgb}{0.33,0.34,0.32}
\definecolor{Aluminium6}{rgb}{0.18,0.20,0.21}
\definecolor{Brown}{cmyk}{0,0.81,1,0.60}
\definecolor{OliveGreen}{cmyk}{0.64,0,0.95,0.40}
\definecolor{CadetBlue}{cmyk}{0.62,0.57,0.23,0}

% ------------------------------------------
% -------- vmargin
% ------------------------------------------

%\setmarginsrb{hleftmargini}{htopmargini}{hrightmargini}{hbottommargini}%{hheadheighti}{hheadsepi}{hfootheighti}{hfootskipi}
\setpapersize{A4}
\setmarginsrb{3cm}{1cm}{3cm}{1cm}{6mm}{7mm}{5mm}{15mm}

% ------------------------------------------
% -------- fancyhdr
% ------------------------------------------
%\fancyheadoffset[L]{\marginparsep+\marginparwidth}
\fancyhf{}
\fancyhead[L]{\bfseries{\nouppercase{\thetitle}}}
\fancyhead[R]{\bfseries{Seite \thepage\ von \pageref{LastPage}}}
\renewcommand{\headrulewidth}{0.5pt}
\renewcommand{\footrulewidth}{0pt}
\fancypagestyle{plain}{
\fancyhf{}
\fancyfoot[R]{\bfseries{Seite \thepage\ von \pageref{LastPage}}}
\renewcommand{\headrulewidth}{0pt}
\renewcommand{\footrulewidth}{0pt}
}

% ------------------------------------------
% -------- hyperref
% ------------------------------------------

\hypersetup{
%breaklinks=true,
pdfborder={0 0 0},
bookmarks=true, % show bookmarks bar?
unicode=false, % non-Latin characters in Acrobat’s bookmarks
pdftoolbar=true, % show Acrobat’s toolbar?
pdfmenubar=true, % show Acrobat’s menu?
pdffitwindow=true, % window fit to page when opened
pdfstartview={FitH}, % fits the width of the page to the window
pdftitle={Multimedia Datenbanken Übung}, % title
pdfauthor={Huber Bastian, Daniel Watzinger}, % author
    pdfsubject={Übungsblatt}, % subject of the document
    pdfcreator={Team Amazonen}, % creator of the document
    pdfproducer={Team Amazonen}, % producer of the document
    pdfkeywords={VAnalyzer}, % list of keywords
    pdfnewwindow=true, % links in new window
    colorlinks=true, % false: boxed links; true: colored links
    linkcolor=black, % color of internal links
    citecolor=black, % color of links to bibliography
    filecolor=magenta, % color of file links
    urlcolor=DarkSkyBlue % color of external links
}

% ------------------------------------------
% -------- listings
% ------------------------------------------
 
\lstset{
breakautoindent=true,
breakindent=2em,
breaklines=true,
tabsize=4,
frame=blrt,
frameround=tttt,
captionpos=b,
basicstyle=\scriptsize\ttfamily,
keywordstyle={\color{SkyBlue}},
commentstyle={\color{OliveGreen}},
stringstyle={\color{OliveGreen}},
showspaces=false,
%numbers=right,
%numberstyle=\scriptsize,
%stepnumber=1,
%numbersep=5pt,
%showtabs=false
prebreak = \raisebox{0ex}[0ex][0ex]{\ensuremath{\hookleftarrow}},
aboveskip={1.5\baselineskip},
columns=fixed,
upquote=true,
extendedchars=true
}
\fontsize{3mm}{4mm}\selectfont

% ------------------------------------------
% -------- misc
% ------------------------------------------
\newcommand{\bibliographyname}{Bibliography}
\setcounter{secnumdepth}{3}
\setcounter{tocdepth}{3}
\clubpenalty = 10000
\widowpenalty = 10000
\displaywidowpenalty = 10000
\setlength\fboxsep{6pt}
\setlength\fboxrule{1pt}
\renewcommand*\oldstylenums[1]{{\fontfamily{fxlj}\selectfont #1}}
%% Set table margins.
{\renewcommand{\arraystretch}{2}
\renewcommand{\tabcolsep}{0.4cm}}

%\renewcommand{\thesubsection}{\alph{subsection})}
%\newcommand{\mysection}[1]{\section*{#1} \setcounter{subsection}{0}}


\bibliographystyle{alpha}
% ------------------------------------------
% -------- hyphenation rules
% ------------------------------------------
%\hyphenation{net.-semanticmetadata.-lire.-imageanalysis.-mpeg7m net.-semanticmetadata.-lire.-imageanalysis}

\author{Bastian Huber\\(51432) \and Sebastian Rainer\\(50882) \and Daniel Watzinger\\(51746) \and Benedikt Preis \\(53279)}
\title{\textbf{\huge{\thetitle}}\\\large\textsc{Gruppe 13}}
\date{\today}

\begin{document}

\maketitle

\pagestyle{fancy}


\mysection{Aufgabe 1: Color Structure Descriptor}
  \subsection{Fehler der mitgelieferten Klasse}
    Die vorgegebene Implementierung ist nicht nur an einer Stelle fehlerhaft. Nachfolgend werden einige M�ngel aufgez�hlt.
    \begin{lstlisting}[language=Java]
int temp[][] =  new int[(int)height - 1][(int)width - 1];
    \end{lstlisting}
    Das Array wird zu klein initialisiert. Damit fehlt der rechte und untere Rand des Bildes und wird nicht bei der Berechnung ber�cksichtigt.
    
    
    \begin{lstlisting}[language=Java]
int ir[][] = temp;
int ig[][] = temp;
int ib[][] = temp;

int iH[][]    = temp;
int iMax[][]  = temp;
int iMin[][]  = temp;
int iDiff[][] = temp;
int iSum[][]  = temp;

...

for (int ch = 0; ch < (int)height - 1; ch++) {
	for (int cw = 0; cw < (int)width - 1; cw++) {
		ir[ch][cw] = mf.getPixelAt(ch,cw).getComponent(0); // RED
		ig[ch][cw] = mf.getPixelAt(ch,cw).getComponent(1); // GREEN
		ib[ch][cw] = mf.getPixelAt(ch,cw).getComponent(2); // BLUE

		int[] tempHMMD = RGB2HMMD(ir[ch][cw],ig[ch][cw],ib[ch][cw]);
		iH[ch][cw]   = tempHMMD[0];						// H
		iMax[ch][cw] = tempHMMD[1];						// Max
		iMin[ch][cw] = tempHMMD[2]; 						// Min
		iDiff[ch][cw]= tempHMMD[3]; 						// Diff
		iSum[ch][cw] = tempHMMD[4]; 						// Sum
		}
}
    \end{lstlisting}
    Die Implementierung deklariert 8 Arrays (\emph{ir, ig, ib, iH, iMax, iMin, iDiff, iSum}) f�r die RGB-Kan�le und die bei der Konvertierung in den
    \emph{HMMD-ColorSpace} anfallenden Parameter. Diese werden aber nicht neu initialisiert sondern referenzieren alle auf das selbe Array.
    Zuf�llig wird hier also immer der \emph{BLUE}-Wert des Pixels verwendet. Zudem wird das Array an der jeweiligen Stelle 
    mehrmals bei der Zuweisung der \emph{HMMD}-Parameter �berschrieben.
    
    
    \begin{lstlisting}[language=Java]
private float[] reQuantization(float[] colorHistogramTemp) {

	float[] uniformCSD = new float[colorHistogramTemp.length];

	for (int i=0; i < colorHistogramTemp.length; i++)
	{
		// System.out.print(colorHistogramTemp[i] + " ");
		// System.out.println(" --- ");

		if (colorHistogramTemp[i] == 0) uniformCSD[i] = 0; //
		else if (colorHistogramTemp[i] < 0.000000001f) uniformCSD[i] = (int)Math.round( ( ((float)colorHistogramTemp[i] - 0.32f ) / ( 1f - 0.32f ) ) * 140 + (115 - 35 - 35 - 20 - 25 - 1) );	 // (int)Math.round((1f / 0.000000001f) * (float)colorHistogramTemp[i]);
		else if (colorHistogramTemp[i] < 0.037f) uniformCSD[i] = (int)Math.round( ( ((float)colorHistogramTemp[i] - 0.32f ) / ( 1f - 0.32f ) ) * 140 + (115 - 35 - 35 - 20 - 25));
		else if (colorHistogramTemp[i] < 0.08f) uniformCSD[i] = (int)Math.round( ( ((float)colorHistogramTemp[i] - 0.32f ) / ( 1f - 0.32f ) ) * 140 + (115 - 35 - 35 - 20));
		else if (colorHistogramTemp[i] < 0.195f) uniformCSD[i] = (int)Math.round( ( ((float)colorHistogramTemp[i] - 0.32f ) / ( 1f - 0.32f ) ) * 140 + (115 - 35 - 35));
		else if (colorHistogramTemp[i] < 0.32f) uniformCSD[i] = (int)Math.round( ( ((float)colorHistogramTemp[i] - 0.32f ) / ( 1f - 0.32f ) ) * 140 + (115 - 35));
		else if (colorHistogramTemp[i] > 0.32f) uniformCSD[i] = (int)Math.round( ( ((float)colorHistogramTemp[i] - 0.32f ) / ( 1f - 0.32f ) ) * 140 + 115);
		else uniformCSD[i] = (int)Math.round((255f / 1f) * (float)colorHistogramTemp[i]);

	}

	return uniformCSD;
}
	\end{lstlisting}
	Die Methode soll die \emph{non-uniform Quantisation} auf \emph{8-Bit}-Werte der Bin-Werte, wie im Standard vorgeschrieben, sicherstellen.
  Alle Werte werden mit \emph{(1f - 0.32f)} skaliert. Dies ist aber im Allgemeinen nicht richtig. Au�erdem benennt der Autor seine Methode widerspr�chlich mit \emph{reQuantization},
  obwohl diese eher \emph{quantization} hei�en m�sste.
  Die \emph{Requantisierung} (von bereits quantisierten Bin-Werten) kommt lediglich beim Berechnen der Distanz von 
  Deskriptoren mit unterschiedlicher Anzahl an Bins (256, 128, 64, 28) zum Einsatz.
	
	Unter anderem wurde die Berechnung der Bin-Indizes sowie die Umwandlung in den \emph{HMMD-Farbraum} (\mbox{\texttt{diff = (max - min) / 2}})
  falsch implementiert.
  \begin{lstlisting}[language=Java]
if (M == 256) {
  offset = 0;
  m = (int)((iH[yy][xx] / M) * quantTable[offset+subspace] + (iSum[yy][xx] / M) * quantTable[offset+subspace+1]);
}
  \end{lstlisting}
  
  F�r die Berechnung der Bin-Indizes wird der 3-dimensionale \emph{HMMD-Farbraum} partitioniert. Die Nummerierung erfolgt laut Standard zuerst
  von unten nach oben entlang der \emph{Sum-Achse}, anschlie�end orthogonal entlang der \emph{Hue-Achse} und schlie�lich von \emph{subspace}
  (Aufteilung der \emph{Diff-Achse}) zu \emph{subspace}. Dies wurde ebenfalls nicht ber�cksichtigt.
  
  Abschlie�end werden Bilder im Querformat von der vorliegenden Implementierung des \emph{Color Structure Descriptors} nicht unterst�tzt.

\begin{lstlisting}[language=Java]
if (width > height) {
  System.out.println("\nExit - vizir bug: file unsupported -> MediaFrame.getPixelAt");
  System.exit(0);
}
\end{lstlisting}  
  

  Als Konsequenz haben wir uns entschlossen, den \emph{Color Structure Descriptor} streng nach Standard selbst zu implementieren.
    
  \subsection{Eigene Implementierung}
    Unsere eigene Implementierung ist eine vollst�ndige Implementierung des \emph{MPEG-7 Color Structure Descriptor} laut Standard. Sie beinhaltet zus�tzlich
    die sogenannte \emph{Requantisierung}. Sie erm�glicht es zwei Deskriptoren mit unterschiedlicher Anzahl an Bins (256, 128, 64, 32) zu
    vergleichen indem der gr��ere der beiden Deskriptoren durch Zusammenfassen von Bins (\emph{Bin-Unification}) verkleinert wird.
    Die Implementierung ist zudem paralellisiert und laufzeitoptimiert. 
    
    Als Distanzmetrik nutzen wir eine gewichtete Variante der $L^1$-Norm. Sie erzielt sehr gute Ergebnisse auf dem Gebiet der
    \emph{Human Recognition} und des \emph{Body Recognition}\cite{Kriech06}.
$$
  weight_{csd,csd'}(i) = \frac{csd(i) + csd'(i)}{\sum\limits_{k = 1}^N csd(k) + csd'(k)}
$$
$$
  distance(csd,csd') = \sum\limits_{i = 1}^N   weight_{csd,csd'}(i) \cdot |csd(i) - csd'(i)|
$$
  \subsection{Integration der Implementierung in Lire}
    \begin{itemize}
      \item Die Klasse \emph{ColorStructurDescriptorImplementation.java} im Paket \emph{net.semanticmetadata.lire.imageanalysis.mpeg7}
            enth�lt die Implementierung des \emph{Color Structure Descriptors}.
            Desweiteren wurde eine Klasse \emph{ColorStructurDescriptor.java} im Paket \emph{net.semanticmetadata.lire.imageanalysis}
            die das \emph{LireFeature} Interface implementiert erstellt.
            Diese bildet die Schnittstelle zwischen der Implementierung des \emph{MPEG-7 Color Structur Descriptors} und \emph{LIRE}.
            Es wird lediglich der Deskriptor serialisiert und deserialisiert und die Berechnungen delegiert.
      \item Zus�tzlich wurde die Klasse \emph{DocumentBuilderFactory} erweitert, um den Deskriptor innerhalb des Frameworks verf�gbar zu machen.
    \end{itemize}

\mysection{Aufgabe 2: Test Applikation}
  Der Entwurf der GUI richtet sich nach den Anforderungen die in der Aufgabenstellung verlangt sind.
  Dabei teilt sich unser Applikation in GUI und Model. Das Model interagiert mit LIRE indem es einen
  Index basieren auf ausgew�hlten Features f�r eine Menge von Bildern erzeugen kann. Zudem gibt es die M�glichkeit, eine \emph{Query by Example} Suche basierend auf
  einem spezifischen Feature anzusto�en. Die Suchergebnisse werden mitsamt den von \emph{Lire} berechneten Scorewerten angezeigt.
  
\mysection{Aufgabe 3: Bewertung}
Unsere Vorgehensweise bei dieser Aufgabe ist wie folgt:
  \begin{itemize}
    \item Wir nehmen ein Beispielbild und suchen die subjektiv $k=10$ �hnlichsten Bilder
    \item Wir z�hlen relevanten Dokumente (im Bezug auf das Beispielbild)
    \item Dann Werten wir die Ergebnisse des Systems aus f�r $k=10$ Bilder.
  \end{itemize}
  
  \subsection{Bilder 0 - 99.jpg}
    \begin{itemize}
      \item Query By 1.jpg
      \item Relevante Bilder: 88
      \item Subjektive Top 10: 1, 10, 11, 16, 21, 36, 66, 74, 92, 97
    \end{itemize}
    
    \begin{center}
    	\begin{tabular}{l|l|l}
    		                        & Precision           & Recall \\
    		                        \hline
    		ColorLayout             & 0.2                 & 0.0227 \\
    		EdgeHistogram           & 0.3                 & 0.0341 \\
    		ColorStructurDescriptor & 0.3                 & 0.0341 \\
    		ScalableColorDescriptor & 0.2                 & 0.0227 \\
    		AutoColorCorrelogram    & 0.1                 & 0.0114 \\
    		CEED                    & 0.4                 & 0.0455 \\
    		FCTH                    & 0.4                 & 0.0455 \\
    		JCD                     & 0.3                 & 0.0341 \\
    		ColorHistorgram         & 0.1                 & 0.0114 \\
    		Tamura                  & 0.2                 & 0.0227 \\
    		Gabor                   & 0.1                 & 0.0114 \\
    	\end{tabular}
    \end{center}
    
    Somit kann f�r diese Bildmenge festgehalten werden, dass im Bezug auf die �hnlichkeit zu 1.jpg, der CEED und FCTH Deskriptor am besten geeignet ist.
    
  \subsection{Kategorie: Elephanten (500--599)}
    \subsubsection{Query: Elefanten und Wasser (594.jpg)}
    \begin{itemize}
      \item \textbf{25 relevante Dokumente:} 594 504 509 513 519 525 532 535 542 545 555 557 559 560 577 582 583 585 587 588 590 591 594 596 597
      \item \textbf{Color Layout:} 594.jpg 588.jpg 587.jpg 583.jpg 597.jpg 560.jpg 542.jpg 504.jpg 513.jpg 592.jpg 584.jpg 557.jpg 531.jpg 500.jpg 517.jpg 555.jpg 586.jpg 573.jpg 579.jpg 589.jpg 595.jpg 593.jpg 577.jpg 527.jpg 570.jpg
      \item \textbf{Edge Histogram:} 594.jpg 588.jpg 519.jpg 528.jpg 581.jpg 509.jpg 520.jpg 531.jpg 558.jpg 503.jpg 569.jpg 563.jpg 512.jpg 583.jpg 572.jpg 557.jpg 534.jpg 574.jpg 514.jpg 560.jpg 542.jpg 515.jpg 587.jpg 575.jpg 576.jpg 
      \item \textbf{Color Structure:} 594.jpg 588.jpg 587.jpg 583.jpg 597.jpg 542.jpg 555.jpg 560.jpg 592.jpg 586.jpg 504.jpg 589.jpg 595.jpg 506.jpg 539.jpg 582.jpg 584.jpg 581.jpg 538.jpg 593.jpg 517.jpg 524.jpg 526.jpg 579.jpg 516.jpg 
      \item \textbf{Scalable:} 594.jpg 558.jpg 588.jpg 531.jpg 592.jpg 593.jpg 582.jpg 575.jpg 515.jpg 589.jpg 536.jpg 595.jpg 565.jpg 570.jpg 564.jpg 561.jpg 521.jpg 553.jpg 584.jpg 548.jpg 513.jpg 556.jpg 514.jpg 538.jpg 579.jpg 
      \item \textbf{Auto Color Correlogram:} 594.jpg 588.jpg 583.jpg 587.jpg 558.jpg 592.jpg 597.jpg 538.jpg 581.jpg 539.jpg 504.jpg 531.jpg 586.jpg 575.jpg 542.jpg 559.jpg 560.jpg 598.jpg 512.jpg 584.jpg 513.jpg 500.jpg 578.jpg 595.jpg 523.jpg 
      \item \textbf{CEED:} 594.jpg 588.jpg 587.jpg 583.jpg 560.jpg 527.jpg 558.jpg 597.jpg 531.jpg 506.jpg 546.jpg 595.jpg 589.jpg 542.jpg 592.jpg 543.jpg 571.jpg 518.jpg 557.jpg 593.jpg 549.jpg 581.jpg 579.jpg 584.jpg 510.jpg 
      \item \textbf{FCTH:} 594.jpg 588.jpg 587.jpg 527.jpg 560.jpg 597.jpg 506.jpg 583.jpg 539.jpg 504.jpg 542.jpg 558.jpg 531.jpg 592.jpg 517.jpg 510.jpg 595.jpg 546.jpg 586.jpg 555.jpg 589.jpg 557.jpg 579.jpg 549.jpg 536.jpg 
      \item \textbf{JPEG Coefficient Histogram:} 594.jpg 583.jpg 509.jpg 534.jpg 518.jpg 588.jpg 572.jpg 538.jpg 560.jpg 567.jpg 511.jpg 543.jpg 500.jpg 545.jpg 542.jpg 524.jpg 503.jpg 555.jpg 587.jpg 533.jpg 576.jpg 581.jpg 527.jpg 522.jpg 552.jpg 
      \item \textbf{Color Histogram:} 594.jpg 588.jpg 583.jpg 587.jpg 531.jpg 560.jpg 597.jpg 542.jpg 558.jpg 527.jpg 506.jpg 592.jpg 504.jpg 571.jpg 555.jpg 538.jpg 546.jpg 518.jpg 577.jpg 543.jpg 579.jpg 536.jpg 595.jpg 586.jpg 589.jpg 
      \item \textbf{Tamura:} 594.jpg 512.jpg 537.jpg 587.jpg 525.jpg 528.jpg 575.jpg 536.jpg 571.jpg 574.jpg 523.jpg 566.jpg 534.jpg 588.jpg 551.jpg 578.jpg 531.jpg 560.jpg 572.jpg 577.jpg 542.jpg 569.jpg 533.jpg 509.jpg 580.jpg 
      \item \textbf{Gabor:} 594.jpg 556.jpg 501.jpg 515.jpg 596.jpg 585.jpg 526.jpg 546.jpg 504.jpg 543.jpg 536.jpg 582.jpg 547.jpg 532.jpg 540.jpg 552.jpg 527.jpg 534.jpg 578.jpg 569.jpg 581.jpg 575.jpg 503.jpg 525.jpg 513.jpg 
    \end{itemize}
    
    \begin{center}
    	\begin{tabular}{l|l|l}
    		                        & Precision           & Recall \\
    		                        \hline
    		ColorLayout             & 0.2                 & 0.0227 \\
    		EdgeHistogram           & 0.3                 & 0.0341 \\
    		ColorStructurDescriptor & 0.3                 & 0.0341 \\
    		ScalableColorDescriptor & 0.2                 & 0.0227 \\
    		AutoColorCorrelogram    & 0.1                 & 0.0114 \\
    		CEED                    & 0.4                 & 0.0455 \\
    		FCTH                    & 0.4                 & 0.0455 \\
    		JCD                     & 0.3                 & 0.0341 \\
    		ColorHistorgram         & 0.1                 & 0.0114 \\
    		Tamura                  & 0.2                 & 0.0227 \\
    		Gabor                   & 0.1                 & 0.0114 \\
    	\end{tabular}
    \end{center}
    
      \subsubsection{Query: Elefanten und gr�nes Gras (568.jpg)}
    \begin{itemize}
      \item \textbf{19 relevante Dokumente:} 568 564 562 554 551 550 544 541 535 522 521 513 512 511 509 508 507 505 501
    \end{itemize}
    
    \begin{center}
    	\begin{tabular}{l|l|l}
    		                        & Precision           & Recall \\
    		                        \hline
    		ColorLayout             & 0.2                 & 0.0227 \\
    		EdgeHistogram           & 0.3                 & 0.0341 \\
    		ColorStructurDescriptor & 0.3                 & 0.0341 \\
    		ScalableColorDescriptor & 0.2                 & 0.0227 \\
    		AutoColorCorrelogram    & 0.1                 & 0.0114 \\
    		CEED                    & 0.4                 & 0.0455 \\
    		FCTH                    & 0.4                 & 0.0455 \\
    		JCD                     & 0.3                 & 0.0341 \\
    		ColorHistorgram         & 0.1                 & 0.0114 \\
    		Tamura                  & 0.2                 & 0.0227 \\
    		Gabor                   & 0.1                 & 0.0114 \\
    	\end{tabular}
    \end{center}

\bibliography{bib}

\end{document}
